\documentclass[a4paper, 11pt]{article}

\usepackage[czech]{babel}
\usepackage[utf8]{inputenc}
\usepackage[left=2cm, top=3cm, text={17cm, 24cm}]{geometry}
\usepackage{times}
\usepackage{graphicx} % vkládání obrázků
\usepackage[unicode]{hyperref}
\usepackage{amsmath}

\usepackage{multirow}


\begin{document}
	\begin{titlepage}
		\begin{center}
			\includegraphics[width=0.77\linewidth]{pics/FIT_logo.pdf} \\

			\vspace{\stretch{0.382}}

			\Huge{Projektová dokumentace} \\
			\LARGE{\textbf{IDS: Finální schéma databáze \uv{Pivovárníci}}} \\
			\vspace{\stretch{0.618}}
		\end{center}

		\begin{minipage}{0.4 \textwidth}
			{\Large \today}
		\end{minipage}
		\hfill
		\Large
		\begin{tabular}{l l}
			\textbf{Jiří Křištof} & \textbf{(xkrist22)} \\
			Petr Češka & (xceska05) \\
		\end{tabular}
	\end{titlepage}
	\newpage
	
\section{Úvod}
Projekt se skládá celkem ze 4 částí

\begin{itemize}
	\item UC a ER diagram
	\item SQL skript vytvářející základní objekty databáze
	\item SQL dotazy nad databází
	\item SQL skript vytvářející pokročilé objekty databáze
\end{itemize}

\section{SQL skript vytvářející základní objekty databáze}
SQL skript z druhé části vytváří tabulky databáze, které propojuje s užitím primárních a cizích klíčů. Schéma databáze (ER diagram) je uvedeno v dokumentu vytvořeném v první části projektu.


Databáze je naplněna ukázkovými daty.


\section{SQL skript vytvářející pokročilé objekty databáze}
Z pokročilých objektů jsou vytvořeny tyto – 2 triggery, 2 procedury, index optimalizující dotaz s pomocí \texttt{EXPLAIN PLAN} a materializovaný pohled. Jsou nastaveny přístupová práva pro druhého člena týmu.

\subsection{Triggery}
Skript obsahuje 2 databázové triggery – \texttt{generate\_account\_id} a \texttt{convert\_to\_euro}.


První zmíněný trigger generuje primární klíč při vkládání do tabulky \texttt{account}, pokud v příkazu insert není explicitně primární klíč uveden.


Druhý trigger při vkládání do tabulky \texttt{offer} převádí cenu piva v nabídce (uvedenou v měně CZK) do měny EUR. 


Funkčnost triggerů je demonstrována na vkládání dat do databáze. 

\subsection{Procedury}
Ve skriptu jsou implementovány 2 procedury – \texttt{beer\_stat} a \texttt{database\_stat}. 
Skript produkuje výstup do konzole pomocí příkazu \texttt{DBMS\_OUTPUT.PUT\_LINE} pro demonstraci chování procedury. 


Procedura \texttt{beer\_stat} vypíše statistiky o pivech – průměrné hodnocení chuti, pěny a vůně a průměrnou cenu piva. 


Procedura \texttt{database\_stat} vypisuje statistiky celé databáze – počet účtů, pivovarů, počet celkových provedených hodnocení, počet vložených piv a existujících pivovarů. 

\subsection{Optimalizace dotazu}

Optimalizace dotazů je dosažena přidáním indexu k sloupci \texttt{type} v tabulce \texttt{beer} a použitím jiného spojení. Pro vypisování průběhu dotazu a statistik jednotlivých kroků je použit příkaz \texttt(EXPLAIN PLAIN).  \\ \\
Pro demonstraci je vybrán dotaz, který vypisuje, kolik tmavých piv používá určité skupenství kvasnic.\\
Po vypsání průběhu neoptimalizovaného dotazu je přidán index \texttt{index\_beer\_type}. Ten zajistí, že když budeme příště přistupovat k některým záznamům ve sloupci \texttt{type}, proběhne rychleji. Při demonstraci lze vidět, že se snížil celkový \texttt{cost} dotazu. Dále změnou spojení tabulek na \texttt{RIGHT JOIN} a přidáním pravidla \texttt{WHERE yeast\_id IS NOT NULL} dosáhneme stejného výstupu, ale umožníme interním operacím (v našem případě \texttt{unique scan}) snížit \texttt{cost} a množství přenesených bytů.
\subsection{Materializovaný pohled}
%TODO

\end{document}
	
	